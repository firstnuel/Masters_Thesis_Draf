%%%%%%%%%%%%%%%%%%%%%%%%%%%%
%%%%%INTRODUCTION%%%%%%%%%%%
%%%%%%%%%%%%%%%%%%%%%%%%%%%%
\section{Introduction}\label{chapter:introduction}

In the thesis we follow the style introduced by The American Psychological Association (APA). The APA style can be found easily in the Internet and some sites provide a quick guide, too. E.g. \url{http://www.waikato.ac.nz/library/learning/g_apaguide.shtml} and \url{http://owl.english.purdue.edu/owl/resource/560/01/} are useful links.

It is important to follow given instructions. In academic theses, not only the content but also the format is important. Generally every academic publication forum requires that the publications follow their guidelines. In the theses accepted in the Department of Information Processing Science the format is APA. Currently there are several editions published from APA. The general rule is that the latest available edition is applied. Currently the newest edition is 6th. If a thesis is already in process it is not needed to transfer it into a newer edition of APA. Whichever you apply, do it consistently.

In addition to teach the students to follow given formal instructions, the guideline aims to unify and standardise the outlook of the theses made in the department. The guideline also enables the supervisors to focus on the content of the theses as the students already consider the outlook and format themselves. In this sense, it is a question of available resources for supervision and guidance.

The use of language and grammar cannot be discussed in detail in this kind of guide. However, the writing style should meet the general academic writing styles in the sense that no causeries are accepted or other lightweight texts such as jokes or rumblings. In other words, in academic theses all writing must be appropriate and reasonable. There are several guidebooks for academic writing available in the Oulu University Library, for example, and in the Internet. For those who write their thesis in Finnish there are books such as Tieteellinen kirjoittaminen. The style reference by APA (American Psychological Association, 2010) offers fruitful practical hints for writing thesis in English.

As the guideline is written according to the instructions, it enables the students to copy their text (without format) on the document and thus get their text into the right format. The format is to be used in the Bachelor’s Theses and in the Master’s Theses. In case of other theses, essays or reports it is recommended that the students inquire their teachers if the guideline is to be followed or not.

The structure of the guideline is as follows. The formal instructions for different topics are presented next. This is followed by examples of references and their use. After that the structure of theses and its writing style is discussed briefly. The guideline ends with a summary.
